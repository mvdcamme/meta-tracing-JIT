\documentclass{beamer}

\usepackage{graphicx,multicol}
%\usepackage{xcolor}
\usepackage{textcomp}
\usepackage{beamerouterthememiniframes, beamercolorthemeann,srcltx,hyperref}
\usepackage{dashrule}
\usepackage{listings}
\usepackage {mathpartir}
\usepackage{pstricks}

\setbeamercolor{normal text}{fg=black!70}
\setbeamertemplate{navigation symbols}{}%geen navigatie
\setbeamertemplate{blocks}[rounded][shadow=true]
\setbeamertemplate{footline}{%
\hspace*{-0.5cm} \raisebox{5pt}{\makebox[\paperwidth]{\hfill\makebox[10pt]{\scriptsize\insertframenumber}}}}
 
\setbeamertemplate{caption}{\raggedright\insertcaption\par}

\title{The essence of meta-tracing JIT compilers}
\author{Maarten Vandercammen \\
		Advisors: Theo D'Hondt, Coen De Roover, \\
				  Joeri De Koster, Stefan Marr and Jens Nicolay}
\date{}

\lstset{
  language=Scheme
}

\definecolor{faded_green}{rgb}{0.56, 0.5, 0.62}

\begin{document}

\begin{frame}[plain]
\includegraphics[width=0.4\paperwidth]{VUB_logo.jpg}
\vspace{2cm}
\titlepage
\end{frame}

\section{Introduction}

\begin{frame}[fragile]{What is tracing JIT compilation?}

\hspace{0.45cm}
\begin{columns}[c]
\column{0.45 \textwidth}
	\begin{lstlisting}[basicstyle=\footnotesize]
	(define (fac x)
 	 (if (< x 2)
 	   1
  	  (* x
   	    (fac (- x 1))))) 
	\end{lstlisting}
	
\column{0.2 \textwidth}
\includegraphics[scale=0.35]{blue_arrow}

\column{0.4 \textwidth}
	\begin{lstlisting}[basicstyle = \scriptsize\ttfamily, escapechar = £]
	(label 'fac-loop)
(literal-value 2)
(save-val)
(lookup-var 'x)
(save-val)
(lookup-var '<)
(apply-native 2)
£\scriptsize{\ttfamily{{\color<2>{red}{(guard-false)}}}}£
(literal-value 1)
(save-val)
(lookup-var 'x)
(save-val)
(lookup-var '-)
(apply-native 2)
...
(goto 'fac-loop)
	\end{lstlisting}
\end{columns}

\pause
Traces are always linear! \\
Use guards to protect against changes in control-flow

\end{frame}

\begin{frame}{Meta-tracing}
\centering
\includegraphics[scale = 0.5]{interpreters_tower}
\end{frame}

\begin{frame}[fragile]{Meta-tracing}
\begin{itemize}

\item Advantage: language implementer can get tracing for free

\pause
%\item Place annotations inside interpreter, e.g. to signal user-loops

\begin{lstlisting}[basicstyle = \scriptsize\ttfamily, escapechar = £]
	(define (close parameters expressions)
      (define lexical-environment environment)
      (define (closure . arguments)
        (define dynamic-environment environment)
        £\scriptsize\ttfamily{\color{red}(can-start-loop expressions)}£
        (set! environment lexical-environment)
        (bind-parameters parameters arguments)
        (let* ((value (evaluate-sequence expressions)))
          (set! environment dynamic-environment)
          value))
      closure)
\end{lstlisting}

\pause
\item PyPy and RPython project \only<3->\footnote{\tiny{Bolz, C. F. (2012)}}

\end{itemize}
\end{frame}

\section{Problem statement}

\begin{frame}{Goal}
\centering
\vspace{1cm}
\Large Construct minimalistic meta-tracing JIT\\
\vspace{1cm}
\includegraphics[scale = 0.35]{interaction_original}
\end{frame}

\begin{frame}{Goal}
\centering
\vspace{1cm}
\Large Grow meta-tracing JIT\\
\vspace{1cm}
\includegraphics[scale = 0.35]{interaction}
\end{frame}

%\begin{frame}{Specify formal semantics}
%Tracing framework
%$\downarrow$
%\begin{mathpar}
% \inferrule*
%  { cesk $\rightarrow$ <cesk', $tau_{2}$> }
%  { <$x_{1}$, $x_{2}$, ..., $\tau_{1}$, cesk> $\rightarrow$ <$x_{1}$, $x_{2}$, ..., $\tau_{1}tau_{2}$, cesk'> }
%\end{mathpar}
%\end{frame}

\begin{frame}{Tracing JIT as state-machine}
\includegraphics[scale = 0.5]{high_level_state_diagram}
\end{frame}

\begin{frame}{Specify formal semantics}
\centering
\Large Tracing framework \\
\vspace{0.5cm}
\includegraphics[scale=0.38]{blue_arrow_down.png} 
\begin{mathpar}
 \inferrule*
  { CESK \rightarrow {\tt <}CESK', \tau_{2}{\tt >} }
  { {\tt <}x_{1}, x_{2}, ..., \tau_{1}, CESK{\tt >} \rightarrow {\tt <}x_{1}, x_{2}, ..., \tau_{1}\tau_{2}, CESK'{\tt >} }
\end{mathpar}
\end{frame}

\section{Roadmap}

\begin{frame}{Roadmap}
\begin{itemize}
\item Literature study {\color{green} \checkmark}
\item Tracing core {\color{green} \checkmark}
\item Additional features
	\begin{itemize}
	\item Trace jumping {\color{green} \checkmark}
	\item Trace merging {\color{faded_green} \checkmark}
	\item True vs. false loops {\color{faded_green} \checkmark}
	\end{itemize}
\end{itemize}
\end{frame}

%\begin{frame}{Refactor again}
%\pspicture(8cm,5cm)
%        \rput(5cm,2cm){\includegraphics[scale = 0.3]{interaction_todo}}
%        \pause
%        \rput(9cm,2cm)%
%        {\fbow{\parbox{6cm}{\bf Specify formal semantics}}}
%        \endpspicture
%\end{frame}

\begin{frame}{Roadmap}
\begin{itemize}
\item Transform state-machine into working operational semantics
\item Specify formal semantics
\item Write thesis
\end{itemize}
\end{frame}

\end{document}